Eine Gruppe von fünf heterosexuellen Paaren geht zusammen ins Musicaltheater.
Dafür wurden zehn Plätze in einer Reihe reserviert.
\begin{teilaufgaben}
\item
Wie viele Möglichkeiten gibt es, wenn jede Person auf irgendeinem
der zehn Plätze sitzen kann?
\item 
Wie gross ist die Wahrscheinlichkeit, dass alle Frauen und alle Männer
geschlossen nebeneinander sitzen?
\item
Bestimmen Sie die Anzahl Möglichkeiten, wenn jeder neben seiner Partnerin
bzw.~neben ihrem Partner Platz nimmt.
\item
Zwei Frauen in dieser Runde vertragen sich momentan nicht besonders gut.
Geben Sie die Wahrscheinlichkeit dafür an, dass die beiden nicht 
nebeneinander sitzen, wenn die Gruppe wie in a) sitzen
darf.
\end{teilaufgaben}

\thema{Kombinatorik}

\begin{loesung}
\begin{teilaufgaben}
\item
Es geht um die Permutationen von $10$ Objekten, also $10!=3628800$.
\item
Es gibt zwei Möglichkeiten, entweder sitzen die Frauen links oder rechts.
Es gibt jeweils $5!$ Möglichkeiten, wie die Frauen bzw.~Männer innerhalb
ihrer Fünfergruppe angeordnet sein können. Die Gesamtzahl der möglichen
Anordnungen als geschlossene Gruppen ist $2\cdot 5!\cdot 5!$.
Die Wahrscheinlichkeit dafür ist daher
\[
P(\text{geschlossene Gruppen})
=
\frac{2\cdot 5!\cdot 5!}{10!}
=
\frac{2}{\binom{10}{5}}
=\frac2{252}=0.0079365.
\]
\item
Fünf Paare müssen platziert werden, dies ist auf $5!=120$ Arten
möglich.
Innerhalb jedes Paares gibt es zwei mögliche Anordnungen\footnote{Wir nehmen
hierbei an, dass die Paare nichts über den Knigge wissen, der verlangt,
dass die Damen immer rechts der Herren Platz zu nehmen haben.}, die
Gesamtzahl der Anordnungen ist also
\[
5!\cdot 2^5=3840.
\]
\item
Zu jeder zulässigen Platzierung der beiden verfeindeten Frauen gibt es
$8!$ mögliche Platzierung der anderen Teilnehmern.
Es gibt nur $9$ Auswahlen von benachbarten Plätzen, und die beiden
Frauen können sich in je zwei Reihenfolgen auf diese beiden Platze
setzen.
Es gibt also $2\cdot 9$ Platzierungen
der beiden Frauen, die führen zu Konflikten,
aber insgesamt $9\cdot 10$ mögliche Platzierungen.
Die Zahl der konfliktfreien Platzierungen ist daher
\[
10\cdot 9-2\cdot 9= 8\cdot 9 = 72.
\]
Die Gesamtzahl der zulässigen Anordnungen ist $72\cdot 8!$.
Die Wahrscheinlichkeit für eine konfliktfreie Anordnung ist daher
\[
P(\text{konfliktfrei}) = \frac{72\cdot 8!}{10!}=\frac{72}{9\cdot 10}
=\frac{8}{10}=0.8.
\qedhere
\]
\end{teilaufgaben}
\end{loesung}

\begin{diskussion}
Aus der Berufsmaturaprüfung 2015 an der Berufsmittelschule Pfäffikon.
\end{diskussion}

\begin{bewertung}
\begin{teilaufgaben}
\item Permutationen ({\bf F}) 1 Punkt,
\item Anzahl Anordnungen als geschlossene Gruppen ({\bf G}) 1 Punkt,
Wahrscheinlichkeit ({\bf W}) 1 Punkt,
\item Anzahl Platzierungen als Paare ({\bf P}) 1 Punkt,
\item Anzahl der konfliktfreien Platzierungen ({\bf K}) 1 Punkt,
Wahrscheinlichkeit ({\bf WK}) 1 Punkt.
\end{teilaufgaben}
\end{bewertung}

